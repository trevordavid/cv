\section{\large COLLOQUIA \& INVITED TALKS}
\begin{tabular}{lll}
{\sc  Date}  & {\sc  Event} & {\sc  Venue} \\
08/2022 & NASA Small Explorer Mission Concept Workshop & NASA JPL, Pasadena, CA\\
07/2021 & Sagan Exoplanet Summer Workshop & NASA Exoplanet Science Institute, Pasadena, CA \\ %Observations of young planets at different evolutionary stages
02/2021 & Center for Exoplanets and Habitable Worlds Seminar & Penn State Univ., State College, PA \\ %Exoplanetary evolution
04/2020 & Winn Exoplanet Group Meeting & Princeton Univ., Princeton, NJ\\ %Exoplanetary evolution
02/2020 & Astrophysics Seminar & AMNH, New York, NY\\ %Exoplanetary evolution
11/2019 & ITC Stars \& Planets Seminar &  Harvard-Smithsonian CfA, Cambridge, MA\\ %Four newborn planets transiting the young solar analog V1298 Tau
03/2019 & Exoplanetary Science Initiative Symposium & Caltech, Pasadena, CA \\ % A warm Jupiter-sized planet transiting the pre-main sequence star V1298 Tau
03/2019 & Astrophysics Seminar & Carnegie Observatories, Pasadena, CA \\ %Temporal evolution of exoplanet demographics
10/2018 & Astronomy Department Colloquium & 
Universit\'{e} de Montr\'{e}al, QC \\ %Towards measuring temporal trends in exoplanet properties and occurrence
09/2018 & ExSoCal 2018 & Caltech, Pasadena, CA \\ %Towards measuring temporal trends in exoplanet properties and occurrence
08/2018 & Exoplanets Seminar & NASA JPL, Pasadena, CA \\ %Towards measuring temporal trends in exoplanet properties and occurrence
05/2018 & Yuk Yung Seminar & Caltech, Pasadena, CA \\%Towards measuring temporal trends in exoplanet properties and occurrence
03/2018 & Exoplanetary Science Initiative Symposium & NASA JPL, Pasadena, CA \\ 
01/2018 & K2 Dwarf Stars and Clusters Workshop & Boston University, Boston, MA \\ %Towards measuring temporal trends in planet occurrence
09/2016 & Center for Integrative Planetary Science Seminar & Univ. of California, Berkeley, CA \\ %Observational constraints on planet formation and migration timescales
04/2016 & K2 CHAI Collaboration Meeting & Univ. of Hawai'i at M\={a}noa, Honolulu, HI \\ %A Neptune-sized transiting planet closely orbiting a 5--10 million-year-old star 
\end{tabular}